\section{Biblioteka}

Biblioteka jest stanowi zbiór funkcji wspomagających komunikację z grą oraz tworzenie tabel i wykresów. Funkcje te działają zarówno w środowisku Matlab jak i Octave.

\paragraph{SendData} \hspace{0pt} \\
Sending data to the server.

%\subparagraph{Syntax}\hspace{0pt} \\
\begin{lstlisting}[style=Matlab-editor]
txt = SendData(IPAddressSend,portSend,data,name, args)
\end{lstlisting}

Description:
\begin{itemize}
\item  IPAddressSend -- server ip address,
\item  portSend -- server port,
\item  data -- data packet sent to the server,
\item  name  -- control information sent to the server,
\item  args  -- arguments related to control information.
\end{itemize}

Returns the response from the server in xml format.

\paragraph{NumberToName} \hspace{0pt} \\
Zamienia tablicę dwuwymiarową w tablicę struktur. Replacing numbers representing field types with their names.
\begin{lstlisting}[style=Matlab-editor]
result = NumberToName(array, names)
\end{lstlisting}

Description:
\begin{itemize}
\item  array -- a table containing information about the map,
\item  names -- map field names.
\end{itemize}

Returns a table containing information about the map.

\paragraph{ChangeBeginEnd} \hspace{0pt} \\
Marking the beginnings and ends of paths.
\begin{lstlisting}[style=Matlab-editor]
result = ChangeBeginEnd(array)
\end{lstlisting}

Description:
\begin{itemize}
\item  array -- a table containing information about the map.
\end{itemize}

Returns a table containing information about the map.

\paragraph{TilemapToXML} \hspace{0pt} \\
Map conversion from an array to xml format.
\begin{lstlisting}[style=Matlab-editor]
txt = TilemapToXML(tilemap)
\end{lstlisting}

Description:
\begin{itemize}
\item  tilemap -- map in the form of an array.
\end{itemize}

Returns a map saved in xml format.

Przykład przesłania do gry polecenia utworzenia planszy:
\begin{lstlisting}[style=Matlab-editor]
%Adres serwera
IPAddressSend = '127.0.0.1';
%Port na ktorym nasluchuje serwer
portSend = 55001;
%Tablica reprezentujaca plansze gry w ktorej liczby reprezentuja typy tiles (1 - ziemia, 2 - woda bedaca elementem sciezki ruchu przeciwnikow, 3 - poczatek sciezki, 4 - koniec sciezki) 
tilemap = [1	3	1	3	1	1	1	1	1	1	1	3	1	1	1 1;
           	1	2	1	2	1	1	1	1	1	1	1	2	1	1	1 1;
           	1	2	1	2	1	1	1	1	3	1	2	2	1	1	1 1;
           	1	2	1	2	1	1	1	1	2	1	2	1	1	1	1 1;
           	1	2	1	2	1	1	1	1	2	1	2	1	1	1	1 1;
           	1	2	1	2	2	1	1	1	2	1	2	2	1	1	1 1;
           	1	2	1	1	2	1	1	2	2	1	1	2	1	1	1 1;
           	1	2	1	2	2	1	1	2	1	1	2	2	1	1	1 1;
           	1	2	1	2	1	1	1	2	1	1	2	1	1	1	1 1;
           	1	2	1	2	1	1	1	2	2	1	2	2	2	2	1 1;
           	1	2	1	2	1	1	1	1	2	1	1	1	1	2	2 1;
           	1	2	1	2	2	1	1	1	2	2	4	1	1	1	2 1;
           	1	4	1	1	4	1	1	1	1	1	1	1	1	1	4 1];
%Nazwy typow tiles. Pozycja w tablicy odpowiada numerowi z tablicy tilemap
names{1} = 'Ground';
names{2} = 'Water';
names{3} = 'Begin';
names{4} = 'End';
%Obrot tablicy tak aby orientacja tablicy odpowiadala orientacji planszy w grze
tilemap = rot90(rot90(rot90(tilemap)));
% Zamiana tablicy na tablice struktur z nazwami tiles zamiast numerow
tilemapNames = NumberToName(tilemap,names);
%Zamiana nazw tiles Begin i End na Water. Przypisanie tym tiles oznaczenia poczatku lub konca sciezki.
tilemapNames = ChangeBeginEnd(tilemapNames);
%Zamiana tablicy sturktur na tekst w formacie xml.
txt = TilemapToXML(tilemapNames);
%Wyslanie polecenia utworzenia nowej planszy (Tilemap) oraz danych w formacie xml do gry.
SendData(IPAddressSend,portSend,txt,'Tilemap',[]);
\end{lstlisting}

\paragraph{ParseXML} \hspace{0pt} \\
Parses text containing xml.
\begin{lstlisting}[style=Matlab-editor]
result = ParseXML(data)
\end{lstlisting}

Description:
\begin{itemize}
\item data -- a text array containing data in xml format.
\end{itemize}

Returns an array of structures whose structure reflects the structure of the XML data, the field names are the names of the XML elements.

Przykład odczytania i zdekodowania pliku xml:
\begin{lstlisting}[style=Matlab-editor]
%Odczytanie pliku xml
dataTower = fileread('towers.xml');
%Konwersja pliku xml
dataTower = ParseXML(dataTower);
\end{lstlisting}

Zawartość pliku towers.xml:
\lstinputlisting[language=XML]{xml/towersBib.xml}

Plik towers.xml zawiera współrzędne wież i ich numery porządkowe. Przykład dostępu do tych danych:
\begin{lstlisting}[style=Matlab-editor]
x=dataTower.Answer.TowerCoordinates{1}.Element{i}.x;
y=dataTower.Answer.TowerCoordinates{1}.Element{i}.y;
no=dataTower.Answer.TowerCoordinates{1}.Element{i}.no;
\end{lstlisting}

\paragraph{GetVectorFromCell} \hspace{0pt} \\
Reading a data vector from a selected field of the structure array.
\begin{lstlisting}[style=Matlab-editor]
res = GetVectorFromCell(data, field)
\end{lstlisting}

Description:
\begin{itemize}
\item data -- structure array,
\item field -- read structure fields.
\end{itemize}

Returns a data vector.

Przykład odczytania współrzędnych x wież jako tablicy:
\begin{lstlisting}[style=Matlab-editor]
%Odczytanie pliku xml
dataTower = fileread('towers.xml');
%Konwersja pliku xml
dataTower = ParseXML(dataTower);
%Odczytanie wspolrzednych x wiez jako tablicy
x = GetVectorFromCell(dataTower.Answer.TowerCoordinates{1}.Element,'x');
\end{lstlisting}

Zawartość pliku towers.xml:
\lstinputlisting[language=XML]{xml/towersBib.xml}

\paragraph{SetEnemies} \hspace{0pt} \\
Creating information in the form of XML about a specific type of opponent.
\begin{lstlisting}[style=Matlab-editor]
txt = SetEnemies(count,speed,startHealth,armour,cost,destroyCoins,coinsToEnd,type,tag)
\end{lstlisting}

Description:
\begin{itemize}
\item count -- maximum number of opponents,
\item speed -- opponent's speed,
\item startHealth -- opponent's starting life value,
\item armour -- enemy's armor (bullet resistance),
\item cost -- the cost of creating and sending an enemy,
\item destroyCoins -- profit for the tower manager for shooting down an enemy,
\item coinsToEnd -- gain for the opponent's manager if he reaches the end of the path,
\item type -- opponent type,
\item tag -- name of the object type.
\end{itemize}

Returns information saved in xml format.

Przykład przesłania do gry polecenia utworzenia nowego typu przeciwnika:
\begin{lstlisting}[style=Matlab-editor]
%Adres serwera
IPAddressSend = '127.0.0.1';
%Port na ktorym nasluchuje serwer
portSend = 55001;
%Utworzenie danych w formacie xml dotyczacych nowego typu przeciwnika
txt = SetEnemies(-1,2,20,2,30,30,40,'Paper','Enemy');
%Wyslanie polecenia (Command) utworzenia nowego typu przeciwnika (SetEnemies)
SendData(IPAddressSend,portSend,txt,'Command','name="SetEnemies"');
\end{lstlisting}

\paragraph{SetTowers} \hspace{0pt} \\
Creating information in the form of XML about a specific type of tower.
\begin{lstlisting}[style=Matlab-editor]
txt = SetTowers(count,speed,rateOfFire,force,bulletStrength,cost,type,tag)
\end{lstlisting}

Description:
\begin{itemize}
\item count -- maximum number of towers,
\item speed -- the rotation speed of the towers,
\item rateOfFire -- rate of fire towers,
\item force -- turret firing power (determines range),
\item bulletStrength -- turret projectile strength (affects the number of wounds dealt to the enemy),
\item cost -- cost of creating a tower,
\item type -- tower type,
\item tag -- the type of object that the tower will attack.
\end{itemize}

Returns information saved in xml format.

Przykład przesłania do gry polecenia utworzenia nowego typu wieży:
\begin{lstlisting}[style=Matlab-editor]
%Adres serwera
IPAddressSend = '127.0.0.1';
%Port na ktorym nasluchuje serwer
portSend = 55001;
%Utworzenie danych w formacie xml dotyczacych nowego typu wiezy
txt = SetTowers(-10,1000,1,1000,5,10,'Tower','Enemy');
%Wyslanie polecenia (Command) utworzenia nowego typu wiezy (SetTowers)
SendData(IPAddressSend,portSend,txt,'Command','name="SetTowers"');
\end{lstlisting}

\paragraph{StartEnemy} \hspace{0pt} \\

Creating an opponent and sending him out along a selected path.
\begin{lstlisting}[style=Matlab-editor]
txt = StartEnemy(beginNo,endNo)
\end{lstlisting}

Description:
\begin{itemize}
\item beginNo -- starting point number,
\item endNo -- endpoint number.
\end{itemize}

Returns information saved in xml format.

Przykład utworzenia przeciwnika i wysłania go z punktu startowego 1 do punktu końcowego 3: 
\begin{lstlisting}[style=Matlab-editor]
%Adres serwera
IPAddressSend = '127.0.0.1';
%Port na ktorym nasluchuje serwer
portSend = 55001;
%Utworzenie danych w formacie xml zawierajacych informacje o punkcie startowym i koncowym
txt = StartEnemy(1,3);
%Wyslanie polecenia (Command) utworzenia przeciwnika i wyslania go od wskazanego punktu startowego do wskazanego punktu koncowego (StartEnemy)
errorStartEnemy = SendData(IPAddressSend,portSend,txt,'Command','name="StartEnemy"');
\end{lstlisting}

\paragraph{AddTower} \hspace{0pt} \\

Adding a tower.
\begin{lstlisting}[style=Matlab-editor]
txt = AddTower(noTower,x,y)
\end{lstlisting}

Description:
\begin{itemize}
\item noTower -- tower number,
\item x -- x coordinate of the tower,
\item y -- y coordinate of the tower.
\end{itemize}

Returns information saved in xml format.

Przykład dodania wieży o numerze 3 w miejsce o współrzędnych $x=1$, $y = 4$: 
\begin{lstlisting}[style=Matlab-editor]
%Adres serwera
IPAddressSend = '127.0.0.1';
%Port na ktorym nasluchuje serwer
portSend = 55001;
%Utworzenie danych w formacie xml zawierajacych informacje o numerze wiezy i jej wspolrzednych
txt = AddTower(3,1,4);
%Wyslanie polecenia (Command) utworzenia wiezy (AddTower)
errorAddTower = SendData(IPAddressSend,portSend,txt,'Command','name="AddTower"');
\end{lstlisting}

\paragraph{GenerateTabular} \hspace{0pt} \\
Generation of a tabular table.
\begin{lstlisting}[style=Matlab-editor]
GenerateTabular(fileName,data,columnDescriptions,rowDescriptions,rowsBold,decimalPlaces)
\end{lstlisting}

Description:
\begin{itemize}
\item fileName -- name of the file to which the array will be saved,
\item data -- saved array,
\item columnDescriptions -- column descriptions,
\item rowDescriptions -- row descriptions, empty array([]) means no descriptions,
\item rowsBold -- 0 means line descriptions are bold and 1 means bold,
\item decimalPlaces -- number of decimal places.
\end{itemize}

Przykład generowania tablicy: 
\begin{lstlisting}[style=Matlab-editor]
%Tablica
exampleArray = [1 2;
                  3 1;
                  5 2;
                  2 4];
%Opisy kolumn
columnDescriptions={'No','Data 1','Data 2'};
%Opisy wierszy
rowDescriptions={'1','2','3','4'};
%Utworzenie pliku zawierajacego srodowisko tablular
GenerateTabular('array.tex',exampleArray,columnDescriptions,rowDescriptions,0,0);
\end{lstlisting}

Zawartość pliku array.tex:
\lstinputlisting[style=lstStyleLaTeX]{table/array.tex}

Tablicę można dołączyć do pliku Latex-a: 
\begin{lstlisting}[style=lstStyleLaTeX]
\begin{table}
\begin{tabular}{|r|r|r|}
  \hline
  \textbf{No}&  \textbf{Data 1}&  \textbf{Data 2}\\
  \hline
  1& 1& 2\\
  \hline
  2& 3& 1\\
  \hline
  3& 5& 2\\
  \hline
  4& 2& 4\\
  \hline
\end{tabular}

\caption{Wygenerowana tablica}
\end{table}
\end{lstlisting}

Uzyskany efekt przedstawia tablica~\ref{array}.

\begin{table}
\begin{tabular}{|r|r|r|}
  \hline
  \textbf{No}&  \textbf{Data 1}&  \textbf{Data 2}\\
  \hline
  1& 1& 2\\
  \hline
  2& 3& 1\\
  \hline
  3& 5& 2\\
  \hline
  4& 2& 4\\
  \hline
\end{tabular}

\caption{Wygenerowana tablica}
\label{array}
\end{table}

\paragraph{GenerateTikzData} \hspace{0pt} \\
Generating data files for tikz charts.
\begin{lstlisting}[style=Matlab-editor]
GenerateTikzData(fileName,data,columnDescriptions)
\end{lstlisting}

Description:
\begin{itemize}
\item fileName -- name of the file to which the array will be saved,
\item data -- saved array,
\item columnDescriptions -- column descriptions.
\end{itemize}

Przykład generowania danych: 
\begin{lstlisting}[style=Matlab-editor]
%Tablica
exampleArray = [1 2;
                  3 1;
                  5 2;
                  2 4];
%Opisy kolumn
columnDescriptions={'No','D1','D2'};
%Utworzenie pliku zawierajacego dane dla wykresow tikz
GenerateTikzData('array.dat',[[1:size(exampleArray,1)]' exampleArray],columnDescriptions);
\end{lstlisting}

Zawartość pliku array.dat:
\lstinputlisting{fig/array.dat}

Plik array.dat można dołączyć do wykresu tikz-a: 
\lstinputlisting[style=lstStyleLaTeX]{fig/array.tikz}

Uzyskany efekt przedstawia rysunek~\ref{Fig:array}.

\begin{figure}
\begin{tikzpicture}
\begin{axis}[
    title={Example},
    xlabel={No},
    ylabel={Data},
    legend pos=outer north east,
    ymajorgrids=true,
    grid style=dashed,
]

\addplot[
    color=blue,
    mark=square
    ]
    table[x=No,y=D1]
    {fig/array.dat};
\addplot[
    color=red,
    mark=square
    ]
    table[x=No,y=D2]
    {fig/array.dat};

    \legend{Data 1, Data 2}

\end{axis}
\end{tikzpicture}

\caption{Wykres na podstawie danych wygenerowanych przez funkcję GenerateTikzData}
\label{Fig:array}
\end{figure}
